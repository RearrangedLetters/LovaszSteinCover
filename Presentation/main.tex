\documentclass[10pt]{beamer}

% Use this to only show the final state of the frames
%\documentclass[handout,10pt]{beamer}

% \input{embed_video.tex}

\usetheme[progressbar=frametitle]{metropolis}
\usepackage{appendixnumberbeamer}

\usepackage{booktabs}
%\usepackage[scale=2]{ccicons}

\usepackage{pgfplots}
\usepgfplotslibrary{dateplot}

%\usepackage[ruled,vlined]{algorithm2e}
\usepackage{amssymb}
\usepackage{gensymb}
\usepackage{xspace}
\usepackage{graphicx}
\usepackage{amsmath}
\newcommand{\themename}{\textbf{\textsc{metropolis}}\xspace}
\newcommand\restr[2]{\ensuremath{\left.#1\right|_{#2}}}

\usepackage{url}
\usepackage{xcolor}

\newcounter{TheoremCounter}
\resetcounteronoverlays{algocf}
\resetcounteronoverlays{TheoremCounter}

\title{Lovász-Stein Theorem}
\subtitle{Seminar Extremal Graph Theory}
\date{20.12.2021}
\institute{Advisors: \newline Prof. Dr. Maria Axenovich \& Dr. Alexander Neal Riasanovsky}
\author{Robin Link}


\begin{document}

\maketitle

\section{Lovász-Stein Theorem}
%\begin{frame}[fragile]{Motivation}
%    \centering{
%        \embedvideo*{\includegraphics[height=.7\textheight]
%                    {Animations/SquareToCirclePreview}}{Animations/SquareToCircle.mp4}
%    }
%\end{frame}


\begin{frame}[fragile]{Motivation}
    \hspace*{-.25em}\includegraphics[width=\textwidth]{Images/Matheson/Matheson_01}
\end{frame}

\begin{frame}[fragile]{Motivation}
    \hspace*{-.25em}\includegraphics[width=\textwidth]{Images/Matheson/Matheson_02}
\end{frame}

\begin{frame}[fragile]{Motivation}
    \hspace*{-.25em}\includegraphics[width=\textwidth]{Images/Matheson/Matheson_03}
\end{frame}

\begin{frame}[fragile]{Motivation}
    \hspace*{-.25em}\includegraphics[width=\textwidth]{Images/Matheson/Matheson_04}
\end{frame}

\begin{frame}[fragile]{Motivation}
    \hspace*{-.25em}\includegraphics[width=\textwidth]{Images/Matheson/Matheson_05}
\end{frame}

\begin{frame}[fragile]{Motivation}
    \hspace*{-.25em}\includegraphics[width=\textwidth]{Images/Matheson/Matheson_06}
\end{frame}

\begin{frame}[fragile]{Motivation}
    \hspace*{-.25em}\includegraphics[width=\textwidth]{Images/Matheson/Matheson_07}
\end{frame}

%%%%%%%%%%%%%%%%%%%%%%%%%%%%%%%%%%%%%%%%%%%%%%%%%%%%%%%%%%%%%%%%%%%%%%%%%%%%%%%%%%%
%                                Greedy sorting                                   %
%%%%%%%%%%%%%%%%%%%%%%%%%%%%%%%%%%%%%%%%%%%%%%%%%%%%%%%%%%%%%%%%%%%%%%%%%%%%%%%%%%%

\begin{frame}[fragile]{Greedy Sorting}
    \begin{center}
        \includegraphics[height=.4\textheight]{Images/Bubblesort/Bubblesort_01}
    \end{center}
\end{frame}

\begin{frame}[fragile]{Greedy Sorting}
    \begin{center}
        \includegraphics[height=.4\textheight]{Images/Bubblesort/Bubblesort_02}
    \end{center}
\end{frame}

\begin{frame}[fragile]{Greedy Sorting}
    \begin{center}
        \includegraphics[height=.4\textheight]{Images/Bubblesort/Bubblesort_03}
    \end{center}
\end{frame}

\begin{frame}[fragile]{Greedy Sorting}
    \begin{center}
        \includegraphics[height=.4\textheight]{Images/Bubblesort/Bubblesort_04}
    \end{center}
\end{frame}

\begin{frame}[fragile]{Greedy Sorting}
    \begin{center}
        \includegraphics[height=.4\textheight]{Images/Bubblesort/Bubblesort_05}
    \end{center}
\end{frame}

\begin{frame}[fragile]{Greedy Sorting}
    \begin{center}
        \includegraphics[height=.4\textheight]{Images/Bubblesort/Bubblesort_06}
    \end{center}
\end{frame}

\begin{frame}[fragile]{Greedy Sorting}
    \begin{center}
        \includegraphics[height=.4\textheight]{Images/Bubblesort/Bubblesort_07}
    \end{center}
\end{frame}

\begin{frame}[fragile]{Greedy Sorting}
    \begin{center}
        \includegraphics[height=.4\textheight]{Images/Bubblesort/Bubblesort_08}
    \end{center}
\end{frame}

\begin{frame}[fragile]{Greedy Sorting}
    \begin{center}
        \includegraphics[height=.4\textheight]{Images/Bubblesort/Bubblesort_09}
    \end{center}
\end{frame}

\begin{frame}[fragile]{Greedy Sorting}
    \begin{center}
        \includegraphics[height=.4\textheight]{Images/Bubblesort/Bubblesort_10}
    \end{center}
\end{frame}

\begin{frame}[fragile]{Greedy Sorting}
    \begin{center}
        \includegraphics[height=.4\textheight]{Images/Bubblesort/Bubblesort_11}
    \end{center}
\end{frame}

%%%%%%%%%%%%%%%%%%%%%%%%%%%%%%%%%%%%%%%%%%%%%%%%%%%%%%%%%%%%%%%%%%%%%%%%%%%%%%%%%%%
%               The second part of the example starts here.                       %
%%%%%%%%%%%%%%%%%%%%%%%%%%%%%%%%%%%%%%%%%%%%%%%%%%%%%%%%%%%%%%%%%%%%%%%%%%%%%%%%%%%

\begin{frame}[fragile]{Lovász-Stein Algorithm}
    \vspace*{-3em}\hspace*{-2em}\includegraphics[width=1.15\textwidth]{Images/LSAExample/00}
\end{frame}

\begin{frame}[fragile]{Lovász-Stein Algorithm}
    \vspace*{-3em}\hspace*{-2em}\includegraphics[width=1.15\textwidth]{Images/LSAExample/01}
\end{frame}

\begin{frame}[fragile]{Lovász-Stein Algorithm}
    \vspace*{-3em}\hspace*{-2em}\includegraphics[width=1.15\textwidth]{Images/LSAExample/02}
\end{frame}

\begin{frame}[fragile]{Lovász-Stein Algorithm}
    \vspace*{-3em}\hspace*{-2em}\includegraphics[width=1.15\textwidth]{Images/LSAExample/03}
\end{frame}

\begin{frame}[fragile]{Lovász-Stein Algorithm}
    \vspace*{-3em}\hspace*{-2em}\includegraphics[width=1.15\textwidth]{Images/LSAExample/04}
\end{frame}

\begin{frame}[fragile]{Lovász-Stein Algorithm}
    \vspace*{-3em}\hspace*{-2em}\includegraphics[width=1.15\textwidth]{Images/LSAExample/05}
\end{frame}

\begin{frame}[fragile]{Lovász-Stein Algorithm}
    \vspace*{-3em}\hspace*{-2em}\includegraphics[width=1.15\textwidth]{Images/LSAExample/06}
\end{frame}

\begin{frame}[fragile]{Lovász-Stein Algorithm}
    \vspace*{-3em}\hspace*{-2em}\includegraphics[width=1.15\textwidth]{Images/LSAExample/07}
\end{frame}

\begin{frame}[fragile]{Lovász-Stein Algorithm}
    \vspace*{-3em}\hspace*{-2em}\includegraphics[width=1.15\textwidth]{Images/LSAExample/08}
\end{frame}

\begin{frame}[fragile]{Lovász-Stein Algorithm}
    \vspace*{-3em}\hspace*{-2em}\includegraphics[width=1.15\textwidth]{Images/LSAExample/09}
\end{frame}

\begin{frame}[fragile]{Lovász-Stein Algorithm}
    \vspace*{-3em}\hspace*{-2em}\includegraphics[width=1.15\textwidth]{Images/LSAExample/10}
\end{frame}

\begin{frame}[fragile]{Lovász-Stein Algorithm}
    \vspace*{-3em}\hspace*{-2em}\includegraphics[width=1.15\textwidth]{Images/LSAExample/11}
\end{frame}

\begin{frame}[fragile]{Lovász-Stein Algorithm}
    \vspace*{-3em}\hspace*{-2em}\includegraphics[width=1.15\textwidth]{Images/LSAExample/12}
\end{frame}


\begin{frame}[fragile]{Memos}
    \textbf{Definitions:} \pause
    \begin{itemize}
        \item $ K $ is the size of the covering found by LSA \pause
        \item $ K_i $ is the number of edges found in step $ i $ \pause
        \item $ a $ is an upper bound for number of $ 1 $'s in a column \pause
        \item $ v $ is a lower bound for the number of $ 1 $'s in a row \pause
    \end{itemize}

    \textbf{Lemmas:}
    \begin{itemize}
        \item[(1)] $ Cov(\mathcal{G}) \leq \sum_{n=1}^a K_i$ \pause
        \item[(2)] $ K_i = \frac{k_{i+1} + k_i}{i} $ \pause
        \item[(3)] $ k_i \leq \frac{(i-1)M}{v}$
    \end{itemize}
\end{frame}



\begin{frame}[fragile]{Lovász-Stein-Theorem}
    \metroset{block=fill}
    \begin{block}{Lovász-Stein Theorem (1974)}
        If each member of $ \mathcal{F} $ has at most $ a $ elements,
        and each point belongs to at least $ v $ of the sets in $ \mathcal{F} $, then
        \[
            Cov(\mathcal{G}) \leq \frac{|\mathcal{F}|}{v}(1 + \ln(a)).
        \]
    \end{block}
\end{frame}

\section{Hashing}

\begin{frame}[fragile]{Hash Functions Motivation}
    How fast can $ x \stackrel{?}{\in} M $ be decided by an algorithm? \pause
    \\
    \vspace{1em}
    Some ideas
    \begin{itemize}
        \item[$ (1) $] For each $ m \in M $ check $ x \stackrel{?}{=} m $.\pause
        \item[$ (2) $] If $ M $ is sortable, all elements can be stored in a binary tree of height $ \log (|M|)$.
        Traverse the tree, checking $ x \stackrel{?}{=} m $ for all visited elements $ m $.\pause
    \end{itemize}
    We get a linear-time algorithm for $ (1) $ and a logarithmic-time algorithm for $ (2) $.\pause
    \\
    \vfill
    \textbf{Question}: How close can we get to constant-time?
\end{frame}

\definecolor{bggray}{HTML}{fafafa}

\begin{frame}[fragile]{Hash Functions}
    \textbf{Idea}:
    \vspace{-.75em}
    \begin{itemize}
        \item Use a function $ h : M \to [n] $, \pause
        \item that can be evaluated in $ \mathcal{O}(1) $ time,\pause
        \item to map elements of $ M $ to positions $ 1, \dots, n $. \pause
    \end{itemize}
    \\
    \vfill

    \textbf{Example} ($ h(x) = x \text{ mod } 5 $):
    \begin{columns}[T]
    \begin{column}{.3\textwidth}
        \includegraphics[height=.5\textheight]{Images/Hashing/Hashing_01}
    \end{column}

    \begin{column}{.5\textwidth}
        \begin{itemize}
            \item[] \semitransp{\textcolor{bggray}{What happens if two items are mapped to the same position?}}
            \item[] \textcolor{bggray}{Time-complexity?}
        \end{itemize}
        \vspace{3em}
        \textcolor{bggray}{$ \to $ Hashing with linear probing on sufficiently large arrays results in
        \textit{expected constant time}.}
    \end{column}
    \end{columns}

    \textcolor{bggray}{\textbf{Question}: Can we get constant-time?}
\end{frame}

\begin{frame}[fragile]{Hash Functions}
    \textbf{Idea}:
    \vspace{-.75em}
    \begin{itemize}
        \item Use a function $ h : M \to [n] $,
        \item that can be evaluated in $ \mathcal{O}(1) $ time,
        \item to map elements of $ M $ to positions $ 1, \dots, n $.
    \end{itemize}
    \\
    \vfill

    \textbf{Example} ($ h(x) = x \text{ mod } 5 $):
    \begin{columns}[T]
    \begin{column}{.3\textwidth}
        \includegraphics[height=.5\textheight]{Images/Hashing/Hashing_02}
    \end{column}

    \begin{column}{.5\textwidth}
        \begin{itemize}
            \item What happens if two items get mapped to the same position? \pause
            \item Time-complexity? \pause
        \end{itemize}
        \vspace{3em}
        $ \to $ Hashing with linear probing on sufficiently large arrays results in
        \textit{expected constant time}. \pause
    \end{column}
    \end{columns}

    \textbf{Question}: Can we get to constant-time?
\end{frame}


\begin{frame}[fragile]{Perfect Hash Functions}
    Let $ \mathcal{F} $ be a family of $ M $-many functions $ f : X \to Y $, where
    $ |X| = n $, $ |Y| = m $. \pause
    
    \textbf{Definition.} The family $ \mathcal{F} $ is a perfect hashing family (PHF), if
    $ \forall C \subseteq X, |C| = w: \exists f \in \mathcal{F}: \restr{f}{C} $ is injective. \pause

    \vfill

    \metroset{block=fill}
    \begin{block}{Theorem 2 (2011)}
        There exists a perfect hashing family with
        \[
            M \leq \frac{n^w}{w!{n \choose w}}\bigg(1 + \ln{m \choose w}\bigg).
        \]
    \end{block}
\end{frame}


\begin{frame}[fragile]{Thank You!}
    \begin{center}
        \large And that's a good place to stop.
        \\
        \vspace*{1em}
        \Large Questions?
    \end{center}
\end{frame}


\appendix
\backupbegin


\begin{frame}[allowframebreaks]{References}
    \nocite{*}
    \bibliographystyle{alpha}
    \bibliography{bibliography}
\end{frame}

\end{document}
