\documentclass[10pt]{beamer}

% Use this to only show the final state of the frames
%\documentclass[handout,10pt]{beamer}

% \input{embed_video.tex}

\usetheme[progressbar=frametitle]{metropolis}
\usepackage{appendixnumberbeamer}

\usepackage{booktabs}
%\usepackage[scale=2]{ccicons}

\usepackage{pgfplots}
\usepgfplotslibrary{dateplot}

%\usepackage[ruled,vlined]{algorithm2e}
\usepackage{amssymb}
\usepackage{gensymb}
\usepackage{xspace}
\usepackage{graphicx}
\usepackage{amsmath}
\newcommand{\themename}{\textbf{\textsc{metropolis}}\xspace}

\usepackage{url}

\newcounter{TheoremCounter}
\resetcounteronoverlays{algocf}
\resetcounteronoverlays{TheoremCounter}

\title{Lovász-Stein Theorem}
\subtitle{Seminar Extremal Graph Theory}
\date{20.12.2021}
\institute{Advisors: \newline Prof. Dr. Maria Axenovich \& Dr. Alexander Neal Riasanovsky}
\author{Robin Link}


\begin{document}

\maketitle

\section{Lovász-Stein Theorem}
%\begin{frame}[fragile]{Motivation}
%    \centering{
%        \embedvideo*{\includegraphics[height=.7\textheight]{Animations/SquareToCirclePreview}}{Animations/SquareToCircle.mp4}
%    }
%\end{frame}

\section{Hashing}

\begin{frame}[fragile]{Hash Functions Motivation}
    How fast can $ x \in M $ be decided by an algorithm?
    \\
    \vspace{1em}
    Some ideas
    \begin{itemize}
        \item[$ (1) $] For each $ m \in M $ check $ x = m $.
        \item[$ (2) $] If $ M $ is sortable, all elements can be stored in a binary tree of height $ \log (|M|)$.
        Traverse the tree, checking $ x = m $ for all visited elements $ m $.
    \end{itemize}
    We get a linear-time algorithm for $ (1) $ and a logarithmic-time algorithm for $ (2) $.
    \\
    \vfill
    \textbf{Question}: How close can we get to constant-time?
\end{frame}


\begin{frame}[fragile]{Hash Functions}
    \textbf{Idea}:
    \vspace{-.75em}
    \begin{itemize}
        \item Use a function $ h : M \to [n] $,
        \item that can be evaluated in $ \mathcal{O}(1) $ time,
        \item to map elements of $ M $ to positions $ 1, \dots, n $.
    \end{itemize}
    \\
    \vfill

    \textbf{Example} ($ h(x) = x \text{ mod } 5 $):
    \begin{columns}[T]
    \begin{column}{.3\textwidth}
        \includegraphics[height=.5\textheight]{Images/HashingExample}
    \end{column}

    \begin{column}{.5\textwidth}
        \begin{itemize}
            \item What happens if two items are mapped to the same position?
            \item Time-complexity?
        \end{itemize}
        \vspace{3em}
        $ \to $ Hashing with linear probing on sufficiently large arrays results in \textit{expected constant time}.
    \end{column}
    \end{columns}

    \textbf{Question}: Can we get constant-time?
\end{frame}


\begin{frame}[fragile]{References}
    \nocite{*}
    \bibliographystyle{alpha}
    \bibliography{bibliography}
\end{frame}

\end{document}
